\documentclass{cheatsheet}
\usepackage{bm}
\usepackage{textcomp, mathcomp}
\usepackage{empheq}
\usepackage{pbox}
\usepackage{booktabs}
\usepackage{verbatim}
\usepackage{graphicx}
\usepackage{tabu}
\usepackage{amssymb}
\usepackage{tikz}

\doctitle{Control Systems I Cheatsheet}
\author{Christian Leser \\ \vspace*{-0.2em}}

\begin{document}
\section{1. Terminology}
	\subsection{1.1 Input, Output, State}
    %\includegraphics[width = \linewidth]{src/images/basic_block_chart.png}
    \begin{itemize}
        \item \textbf{(Control) Input u(t)} (gas pedal)
        \begin{itemize}
            \item endogenous: manipulation by designer
            \item exogenous: generated by environment (e.g. Weather)
        \end{itemize}
        \item \textbf{Output / Measurement y(t)} (speed)
        \begin{itemize}
            \item measured outputs: Quantities that we can measure
            \item performance outputs: unmeasurable but controllable (e.g. average fuel consumption)
        \end{itemize}
        \item \textbf{state x(t)}: "memory", summary of all past inputs (fuel)
        \item parameter: quantities that do not change over time (colour)
\end{itemize}
	\subsection{Classification}
    \textcolor{blue}{L}\textcolor{orange}{TI} \textcolor{teal}{SISO}: \textcolor{blue}{Linear} \textcolor{orange}{Time invariant} \textcolor{teal}{single input single output}
    \subsubsection*{Linear vs. nonlinear Systems}
        For a linear system, Additivity and Homogenity must hold:
        \begin{align*}
            \Sigma (\alpha u_a + \beta u_b) = \alpha (\Sigma u_a) + \beta (\Sigma u_b) = \alpha y_a + \beta y_b
        \end{align*}
        Therefore, superposition holds for linear systems.\\
        Linear System can be represented in state-space-form, the matrices do not depend on time:
        \begin{equation}\label{eqn:LTI}
            \begin{cases}
                \dot{x}(t) = Ax(t) + Bu(t)\\
                y(t) = Cx(t) + Du(t)
            \end{cases}
        \end{equation}

    \subsubsection*{Causal vs non-causal systems}
        Output at time $t$ depends only on the values of input on $(-\infty, t]$ (future inputs do not affect current output)\\
        $y(t) = f(u(\tau)), \tau < t$
    
    \subsubsection*{Static systems}
        Static (memoryless) vs Dynamic
        y(t) only depends on u(t), output does not depend on past or future\\
        $dim(x) = 0$ : dimension of system equals 0

    \subsubsection*{Time invariant vs Time-varying !überarbeiten!}
        output does not depend on time, only on $\Delta t$ (counter example sun clock)\\
        {\centering\textbf{\underline{Time-varying}}}: Matrices $A(t), B(t), C(t), D(t)$ depend on t
%        \begin{equation}
%            t_0 = t_0
%            \begin{cases}
%                \dot{x}(t) = f(t, x(t), u(t))\\
%                y(t) = g(t, x(t), u(t))
%            \end{cases}
%            \rightarrow
%            \begin{cases}
%                \dot{x}(t) = A(t)x(t) + B(t)u(t)\\
%               y(t) = C(t)x(t) + D(t)u(t)
%            \end{cases}
%        \end{equation}

        {\centering\textbf{\underline{Time Invariant}}}
        \begin{align*}
            \text{time shift operator $\sigma_{\tau}$:\quad \quad} \sigma_{\tau} y(t) = \Sigma \sigma_{\tau} u(t)\\
            y(t - \tau) = \Sigma(u(t - \tau))\\
            t_0 = 0
            \begin{cases}
                \dot{x}(t) = f(x(t), u(t))\\
                y(t) = g(x(t), u(t))
            \end{cases}
            \rightarrow
            \begin{cases}
                \dot{x}(t) = Ax(t) + Bu(t)\\
                y(t) = Cx(t) + Du(t)
            \end{cases}
        \end{align*}
	\subsection{Interconnections}
    \includegraphics[width = \linewidth]{src/images/interconnection.png}
%	\subsection{Basic Control Architectures}
    \begin{tabu}{X[m] X[2, c, m]}
        \textbf{Name}               & \textbf{Block chart}\\
        \hline \hline
        Feed-forward      & \includegraphics[width = \linewidth]{src/images/architecture_feed_forward.png}\\
        \hline
        Feedback    & \includegraphics[width = \linewidth]{src/images/architecture_feedback.png}\\
        \hline
        Two degrees of freedom    & \includegraphics[width = \linewidth]{src/images/architecture_two_freedom.png}
    \end{tabu}

\section{2. System Modeling}
	for LTI SISO systems: $\frac{d}{dt} \text{storage} = \sum \text{inflows} - \sum \text{outflows}$

    \begin{center}
        \textbf{\underline{Mass conservation}}    
    \end{center}
    \begin{align*}
        \frac{d}{dt} m = \sum m_{in}(t) - \sum m_{out}(t)
    \end{align*}

    \begin{center}
        \textbf{\underline{Mechanical systems}}    
    \end{center}
    \begin{align*}
        M(t) = J \cdot \ddot{\phi}(t)
    \end{align*}

    \begin{center}
        \textbf{\underline{Thermodynamic systems}}    
    \end{center}
    \begin{align*}
        m \frac{d}{dt} T(t) = c(T_{ext}(t) -T(t)) + u(t)
    \end{align*}
	\subsection{State Space Form}
    \begin{align*}
        \begin{cases}
            \dot{x}(t) = f(x(t), u(t), d(t))\\
            y(t) = f(x(t), u(t), d(t))
        \end{cases}\\
        \text{dim}(x) = \text{Order / Dimension of system}\\ 
        \text{(minimum \# variables sufficient to describe state)}
    \end{align*}
    
	\input{src/2_modeling/3_jacobian.tex}
	\subsection{time response to LTI system}
    \begin{center}
        \textbf{Superposition of responses:}
    \end{center}
    
    \begin{minipage}{0.49\linewidth}
        \begin{minipage}{0.49\linewidth}
            non-zero initial conditions (natural response)
        \end{minipage}
        \begin{minipage}{0.49\linewidth}
            \begin{align*}
                \begin{cases}
                    x(0) = x_0\\
                    u(t) = 0
                \end{cases}
            \end{align*}
        \end{minipage}
    \end{minipage}
    \begin{minipage}{0.49\linewidth}
        \begin{minipage}{0.44\linewidth}
            non-zero inputs (forced response)
        \end{minipage}
        \begin{minipage}{0.54\linewidth}
            \begin{align*}
                \begin{cases}
                    x(0) = 0\\
                    u(t) = u(t)
                \end{cases}
            \end{align*}
        \end{minipage}
    \end{minipage}
    \begin{align*}
        x(t) &= \underbrace{e^{At} \cdot x_0}_{\substack{\text{hom. solution}\\ u(t) = 0}} + \underbrace{\int\limits_0^t e^{A (t-\rho)} B u(\rho) d\rho}_{\substack{\text{inhom. solution, zero initial condition}\\ x(0) = 0}}\\
        y(t) &= \underbrace{C \cdot e^{At} \cdot x_0}_{\substack{\text{natural / initial} \\ \text{condition response}}} + \underbrace{C \cdot \int\limits_0^t e^{A (t-\rho)} B u(\rho) d\rho}_{\text{forced response}} + \underbrace{D u(t)}_{\text{feedthrough}}
    \end{align*}

    state transition function: $\psi(t) = e^{At}$ %Maybe how to get the solution of an lti? see lecture 4 Time response
	\subsection{Stability}
    \includegraphics[width = \linewidth]{src/images/eigenvalue_response.png}
    system behaviour based on eigenvalues of A

    \subsubsection{Lyapunov stable}
    $\lim_{t \rightarrow \infty} x(t) \neq \pm \infty \Leftrightarrow Re(\lambda_i) \leq 0, Re(\lambda_i) = 0 \text{for exactly one i}$

    \subsubsection{Asymptotically stable}
    $\lim_{t \rightarrow \infty} x(t) = 0 \Leftrightarrow Re(\lambda_i) < 0$

    \subsubsection{BIBO (Bounded Input Bounded Output) stability}
    $\lim_{t \rightarrow \infty} y(t) \neq \pm \infty \Leftrightarrow Re(\lambda_i) < 0$

    \begin{center}
        \textbf{Not stabilizable and observable systems}
    \end{center}
    \begin{tabu}{|X[6] X X[3]|}
        \hline
        asymptotically stable & $\rightarrow$ & BIBO stable\\
        ? & $\leftarrow$ & BIBO stable\\
        Lyap. stable or unstable & $\rightarrow$ & ?\\
        Lyap. stable or unstable & $\leftarrow$ & BIBO unstable\\
        \hline
    \end{tabu}

    \subsubsection{Unstable}
    $\lim_{t \rightarrow \infty} x(t) = \pm \infty \Leftrightarrow Re(\lambda_i) > 0 \text{for at least one i}$
	\subsection{Frequency response / domain (time $\rightarrow$ frequency)}
choose $u(t) = e^{st}$ where s is a complex number\\
$ y(t) = \underbrace{Ce^{At} [x(0) - (sI - A)^{-1} B]}_{\text{Transient response (= 0 if as. stable)}}
+ \underbrace{[C(sI - A)^{-1} B + D]e^{st}}_{\text{steady-state response}}$\\
\includegraphics*[width = \linewidth]{src/images/transient_steady_state.png}
$\Rightarrow y_{ss} = G(s) e^{st}\\
G(s) = \frac{Y(s)}{U(s)} = C(sI - A)^{-1} B + D = \frac{C \cdot \text{Adj}(sI - A) \cdot B}{\text{det}(sI - A)} + D$\\
\begin{align*}
    \text{adj}\left[
        \begin{array}{c c}
            a & b\\
            c & d            
        \end{array}
    \right]
    = \left[
        \begin{array}{c c}
            d & -b\\
            -c & a
        \end{array}
    \right]
    \\
    \text{adj}\left[
        \begin{array}{c c c}
            a & b & c\\
            d & e & f\\
            g & h & i
        \end{array}
    \right]
    = \left[
        \begin{array}{c c c}
            ei-fh & ch-bi & bf-ce\\
            fg-di & ai-cg & cd-af\\
            dh-eg & bg-ah & ae-bd            
        \end{array}
    \right]
\end{align*}
	\subsection{Writing the transfer function G(s)}
    \titel{Partial fraction expansion}
        \begin{align*}
            G(s) = \frac{r_1}{s-p_1} + \frac{r_2}{s-p_2} + \ldots + \frac{r_n}{s-p_n} + r_0
        \end{align*}
        $p_1, \ldots, p_n =$ poles, $r_1, \ldots, r_n =$ residues
    \titel{Root-locus}
        \begin{align*}
            G(s) &= \frac{k_{\text{rl}}}{s^q} \frac{(s-z_1)(s-z_2) \dots (s-z_m)}{(s-p_1)(s-p_2) \dots (s-p_{n-q})}
        \end{align*}
    \titel{Bode form}
        \begin{align*}
            G(s) &= \frac{k_{\text{bode}} \cdot k_{\text{rl}}}{s^q} \frac{(\frac{s}{-z_1} + 1)(\frac{s}{-z_2} + 1) \dots (\frac{s}{-z_m} + 1)}{(\frac{s}{-p_1} + 1)(\frac{s}{-p_2} + 1) \dots (\frac{s}{-p_{n-q}} + 1)}\\
            k_{\text{bode}} &= \frac{(-z_1)(-z_2) \dots (-z_m)}{(-p_1)(-p_2) \dots (-p_{n-q})}, k_{\text{bode}} \cdot k_{\text{rl}} = \underbrace{y_{ss}(t) = G(0)}_{\text{for stable systems}}
        \end{align*} 
	\subsection{Compute phase and magnitude of transfer function}
\begin{align*}
    u(t) &= sin(\omega t) \rightarrow y_{ss}(t) = \left|G(j \omega)\right| \sin\left(\omega t + \angle(G(j \omega))\right)\\
    G(s) &= k \frac{(s-z_1)(s-z_2)}{(s-p_1)(s-p_2)}\\
    |G(s)| &= k \frac{|s-z_1| \cdot |s-z_2|}{|s-p_1| \cdot |s-p_2|}\\
    \angle(G(s)) &= \angle(s-z_1) + \angle(s-z_2) - \angle(s-p_1) - \angle(s-p_2)
\end{align*}
	% 9_calculate_poles_and_zeros.tex
	% 10 Visualization in complex plane
	\subsection{controllable canonical form (frequency $\rightarrow$ time)}
\titel{$A$ is diagonal:}
\begin{align*}
    G(s) = \frac{p_1}{s-\lambda_1} + \frac{p_2}{s-\lambda_2} + ... + \frac{p_n}{s-\lambda_n} + d\\
    A = \left[\begin{array}{c c c}
        \lambda_1 & & \\
         & \ddots & \\
         & & \lambda_n
    \end{array}\right]
    ,
    B = \left[\begin{array}{c}
        \sqrt{p_1}\\
        \vdots\\
        \sqrt{p_n}
    \end{array}\right]
    \\
    C = \left[\begin{array}{c c c}
        \sqrt{p_1} & \cdots & \sqrt{p_n}
    \end{array}\right]
    ,
    D = d
\end{align*}

\titel{general case:}
\begin{align*}
    G(s) &= \frac{b_{n-1}s^{n-1} + b_{n-2}s^{n-2} + \dots + b_0}{s^n + a_{n-1}s^{n-1} + \dots + a_0} + d\\
    A &= \left[\begin{array}{c c c c c}
        0 & 1 & 0 & \cdots & 0\\
        0 & 0 & 1 & \cdots & 0\\
        \vdots & \vdots & & \ddots & 1\\
        -a_0 & -a_1 & \cdots & & -a_{n-1}
    \end{array}\right]
    ,
    B = \left[\begin{array}{c}
        0\\
        0\\
        \vdots\\
        1
    \end{array}\right]
    \\
    C &= \left[\begin{array}{c c c c}
        b_0 & b_1 & \cdots & b_{n-1}
    \end{array}\right]
    ,
    D = d
\end{align*}
	\subsection{poles and zeroes}
- complex poles: oscillating system $(\omega \propto Im(p))$\\
- complex poles and zeros: always complex conjugate pairs\\
- stable pole: $Re(p) < 0$\\
- minimum-phase zero: $Re(z) < 0 \rightarrow$ overshoot\\
- minimum-phase zero, $Re(z) = 0$: zero slope at $t_0$\\
- non-minimum-phase zero: $Re(z) > 0 \rightarrow$ undershoot\\
- minimum realizable system $\Leftrightarrow$ No zero-pole cancellation

	%differentiator and integrator (Lecture 5 Page 24, 25, Lecture 6 page 31, 32)
	
\section{System Analysis}
	\input{src/3_analysis/1_root_locus.tex}

\section{System Manipulation}
	\subsection{PID Controller}
    \titel{Differentiator}
        \begin{align*}
            y(t) = \frac{d u(t)}{dt} = \frac{d}{dt} e^{st} = s \cdot e^{st} = s \cdot u(t)
        \end{align*}
    \titel{Integrator}
        \begin{align*}
            y(t) = \int u(t) dt = \int e^{st} dt = \frac{1}{s} \cdot e^{st} = \frac{1}{s} \cdot u(t)
        \end{align*}

    \titel{Building the controller}
        \begin{minipage}{0.49\linewidth}
            \begin{align*}
                C(s) &= k_P + \frac{k_I}{s} + k_D \cdot s\\
                &= \frac{k_P \cdot s + k_I + k_D \cdot s^2}{s}\\
                &= k_P(1 + \frac{1}{T_I \cdot s} + T_D \cdot s)
            \end{align*}
        \end{minipage}
        \begin{minipage}{0.49\linewidth}
            \begin{scriptsize}
                \begin{align*}
                    k_P &= \text{Proportional gain}\\
                    k_I &= \text{Integral gain}\\
                    k_D &= \text{Derivative gain}\\
                    T_I &= \text{Integral time constant gain}\\
                    T_D &= \text{Derivative time constant}
                \end{align*}
            \end{scriptsize}
        \end{minipage}
        %Von ZF: P-controller, I-controller, D-controller
	\subsection{Lead- and Lag compensators}
    \begin{align*}
        C_{\text{lead/lag}} = \frac{\frac{s}{a} + 1}{\frac{s}{b} + 1} = \frac{b}{a} \frac{s+a}{s+b}, \varphi_{\text{max}} = 2 \arctan\left(\sqrt{\frac{b}{a}} - 90^{\circ}\right)\\
        \text{Magnitude in- / decrease } (\omega >> 1): |C(j \omega)| = \pm 20 dB
    \end{align*}
    
    \titel{Lead-Controller}
        \begin{minipage}{0.49\linewidth}
            \begin{align*}
                0 < a < b
            \end{align*}
            Use: increase phase margin\\
            – $k \cdot \sqrt{ab}$: desired crossover frequency\\
            – $\frac{b}{a}$: use for desired phase increase\\
        \end{minipage}
        \begin{minipage}{0.49\linewidth}
            \includegraphics[width = \linewidth]{src/images/lead-controller.png}
        \end{minipage}

    \titel{Lag-Controller}
    \begin{minipage}{0.49\linewidth}
        \begin{align*}
            0 < b < a
        \end{align*}
        Use: improve command tracking / disturbance rejection\\
        To achieve this: multiply $k$ with $\frac{a}{b}$ and choose $a$ big enough not to affect crosover
    \end{minipage}
    \begin{minipage}{0.49\linewidth}
        \includegraphics[width = \linewidth]{src/images/lag-controller.png}
    \end{minipage}

    Show in Bode Plot: Phase in / decreases between $\omega_1 = a$ and $\omega_2 = b$, midpoint between $b$ an $a$ is given by $\sqrt{ab}$ 

    \titel{similarities of PID and Lead / Lag compensators}
        \begin{align*}
            PID(s) = k \cdot \underbrace{\frac{\frac{s}{z_1} + 1}{s + 0}}_{\text{Lag}} \cdot \underbrace{\frac{\frac{s}{z_2} + 1}{\frac{s}{p} + 1}}_{\text{Lead}}
        \end{align*}
        p refers to a "fast pole", This is necessary for the controller to be implementable
	
\section{System Synthesis}

\section{Appendix}
	\subsection{compute matrix exponential}
    \subsubsection{Taylor expansion}
        \begin{align*}
            e^{At} = I + At + \frac{1}{2}(At)^2 + ... + \frac{1}{n!}(At)^n + ...
        \end{align*}
    \subsubsection{Diagonalization}
        \begin{align*}
            A &= TDT^{-1}\\
            \Rightarrow exp(At) &= T e^{Dt} T^{-1}\\
            &= T \text{diag}(e^{\lambda_1 t}, e^{\lambda_2 t}, ..., e^{\lambda_n t}) T^{-1}\\
            T &= (EV_1, EV_2, ..., EV_n)
        \end{align*}
    \subsubsection{Jordan matrix}
        \begin{align*}
            exp\left(
                \left[
                    \begin{array}{c c c}
                        \lambda & 1 & 0\\
                        0 & \lambda & 1\\
                        0 & 0 & \lambda
                    \end{array}
                \right]
                t
            \right)
            =
            \left[
                \begin{array}{c c c}
                    1 & t & \frac{1}{2!} t^2\\
                    0 & 1 & t\\
                    0 & 0 & 1
                \end{array}
            \right]
            e^{\lambda t}
        \end{align*}
	\subsection{Euler Formula}
    \begin{align*}
        &e^{js} = cos(js) + j sin(js)\\
        &cos(s) = \frac{e^{js} + e^{-js}}{2} \quad \quad sin(s) = \frac{e^{js} - e^{-js}}{2j}
    \end{align*}
	\subsection{cover-up method for partial fraction expansion}
    \begin{align*}
        f(x) = \frac{(x-1)(x+5)}{x(x+1)\colorbox{green}{$(x+2)$}} = \frac{A}{x} + \frac{B}{x+1} + \frac{C}{\colorbox{green}{$(x+2)$}}\\
        \colorbox{green}{$(x+2)$} = 0 \rightarrow x = \colorbox{red}{-2} \rightarrow C = \frac{(\colorbox{red}{$(-2)$}-1)(\colorbox{red}{$(-2)$}+5)}{(\colorbox{red}{$(-2)$})(\colorbox{red}{$(-2)$}+1)}
    \end{align*}

	\input{src/5_appendix/4_visualization_poles.tex}

\end{document}

\begin{comment}
	TERMINOLOGY
	Definitions from ZF
Input, output, state…
	Classification of Systems (Also see ZF)
SYSTEM MODELING / transfer real world problems to input output system
	Thermodynamics
	Conservation of mass: for LTI SISO systems: $\frac{d}{dt} \text{storage} = \sum \text{inflows} - \sum \text{outflows}$

	Mechanics
NETWORK ANALYSIS
	(Interconnections & Basic Control Architectures) (Block Diagrams on ZF)
LINEARIZATION
	Jacobian Linearization process
LINEAR SYSTEM ANALYSIS
TIME RESPONSE
STABILITY
	Equilibrium point: $\dot{x}(t) = 0
FREQUENCY DOMAIN
	y(t) = G(s) * u(t) for u(t) = e^(st)
	Therefore, for sinusoidal inputs: u(t) = cos(omega * t) -> y = M cos(omega * t + phi), M = |G(j * omega)|, phi = \angle G(j * omega
	e^(i*s*t) = cos(s*t) + i*sin(s*t), cos(s*t) = 
	STATE SPACE -> TRANSFER FUNCTION
	TRANSFER FUNCTION -> CONTROLLABLE CANONICAL FORM
	POLES/ZEROS
SIMPLE LAPLACE
	derivative of a laplace function
	laplace of step function
	laplace of x^n

	!!! Disturbance and Noise rejection !!!
\end{comment}