\subsection{Classification}
    \textcolor{blue}{L}\textcolor{orange}{TI} \textcolor{teal}{SISO}: \textcolor{blue}{Linear} \textcolor{orange}{Time invariant} \textcolor{teal}{single input single output}
    \subsubsection*{Linear vs. nonlinear Systems}
        For a linear system, Additivity and Homogenity must hold:
        \begin{align*}
            \Sigma (\alpha u_a + \beta u_b) = \alpha (\Sigma u_a) + \beta (\Sigma u_b) = \alpha y_a + \beta y_b
        \end{align*}
        Therefore, superposition holds for linear systems.\\
        Linear System can be represented in state-space-form, the matrices do not depend on time:
        \begin{equation}\label{eqn:LTI}
            \begin{cases}
                \dot{x}(t) = Ax(t) + Bu(t)\\
                y(t) = Cx(t) + Du(t)
            \end{cases}
        \end{equation}

    \subsubsection*{Causal vs non-causal systems}
        Output at time $t$ depends only on the values of input on $(-\infty, t]$ (future inputs do not affect current output)\\
        $y(t) = f(u(\tau)), \tau < t$
    
    \subsubsection*{Static systems}
        Static (memoryless) vs Dynamic
        y(t) only depends on u(t), output does not depend on past or future\\
        $dim(x) = 0$ : dimension of system equals 0

    \subsubsection*{Time invariant vs Time-varying !überarbeiten!}
        output does not depend on time, only on $\Delta t$ (counter example sun clock)\\
        {\centering\textbf{\underline{Time-varying}}}
        \begin{equation}
            t_0 = t_0
            \begin{cases}
                \dot{x}(t) = f(t, x(t), u(t))\\
                y(t) = g(t, x(t), u(t))
            \end{cases}
            \rightarrow
            \begin{cases}
                \dot{x}(t) = A(t)x(t) + B(t)u(t)\\
                y(t) = C(t)x(t) + D(t)u(t)
            \end{cases}
        \end{equation}

        {\centering\textbf{\underline{Time Invariant}}}
        \begin{align*}
            \text{time shift operator $\sigma_{\tau}$:\quad \quad} \sigma_{\tau} y(t) = \Sigma \sigma_{\tau} u(t)\\
            y(t - \tau) = \Sigma(u(t - \tau))\\
            t_0 = 0
            \begin{cases}
                \dot{x}(t) = f(x(t), u(t))\\
                y(t) = g(x(t), u(t))
            \end{cases}
            \rightarrow
            \begin{cases}
                \dot{x}(t) = Ax(t) + Bu(t)\\
                y(t) = Cx(t) + Du(t)
            \end{cases}
        \end{align*}